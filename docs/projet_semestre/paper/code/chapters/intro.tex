% !TeX spellcheck = fr_FR
\chapter*{Introduction}
\addcontentsline{toc}{chapter}{Introduction} % Adding toc entry

% Mise en contexte

Ce travail s'inscrit dans le cadre de mon travail de semestre, réalisé en réponse à une demande d'ELCA Security, une entreprise spécialisée dans la cyber-sécurité. 
Le projet présenté ici vise à explorer une approche peu commune pour attaquer des mots de passe protégés par l'algorithme bcrypt en utilisant un \gls{fpga}. 
L'idée serait de préparer le terrain, pour que par la suite, je puisse reprendre le travail pour mon travail de bachelor.
Ce projet a pour objectif final de leur fournir une solution extensible et performante qui puisse être utilisable lors de leurs attaques. 
L'intérêt technique et scientifique de ce projet réside dans la recherche de solutions moins énergivores pour l'attaque de mots de passe, en tirant parti des capacités de traitement parallèle offert par les \gls{fpga}.\\\

% Méthodologie de travail

Après une recherche approfondie des implémentations existantes du bcrypt sur \gls{fpga}, le projet a débuté par une analyse du papier\footcite{wiemer_high-speed_2014} concernant une implémentation déjà existante. 
En parallèle, une page Wikipédia détaillant le fonctionnement de bcrypt\footcite{noauthor_bcrypt_2024} a été utilisée comme ressource principale pour comprendre les spécificités de cet algorithme. 
De plus, un papier sur une attaque MD5\footcite{gillela_parallelization_2019} sur \gls{fpga} a été consulté pour enrichir la compréhension des techniques d'attaque sur des dispositifs matériels. 
Ces ressources documentaires ont été cruciales pour comprendre les différents concepts, orienter les choix de conception et résoudre les problèmes techniques rencontrés.

Dans le cadre de ce projet, j'ai entrepris plusieurs actions significatives. 
Tout d'abord, j'ai repris le code de l'implémentation existante en VHDL d'un programme d'attaque par bruteforce de mot de passe bcrypt. 
Après avoir étudié le papier associé et constaté des incohérences dans les testbenches, j'ai refait ces derniers pour valider le programme. 
Par la suite, j'ai identifié et corrigé des erreurs dans le code afin d'obtenir un programme fonctionnel.

En parallèle, j'ai pu brièvement étudier le fonctionnement du \gls{pcie} afin d'y mettre en place une simple interface entre une carte \gls{fpga} et un ordinateur.\\\

% Annonce du plan

Dans ce rapport, je vais commencer par brièvement expliquer les différents notions clés de ce projet, afin de poser une base technique. 
Je vais par la suite vous présenter une analyse du projet, afin d'y décrire le projet de manière détaillé, expliquer le fonctionnement du Bcrypt et les recherches qui ont été faites afin de trouver une implémentation existante. 
Puis, je vais détailler la méthodologie de travail, en exposant les différentes étapes de mise en œuvre du projet, les simulations qui ont été faites et les différentes difficultés rencontrées. 
Enfin, je vais décrire les résultats obtenus lors des différents tests et mesures qui ont été faits.
